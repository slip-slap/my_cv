\begin{figure}[!b]
	\centering
		\begin{tikzpicture}[thick, scale=0.5, every node/.style={transform shape}]
			\tikzstyle{startstop} = [rectangle, rounded corners, minimum width=1.0cm,minimum height=0.6cm, text centered, draw=black]
			\tikzstyle{rec} = [rectangle,rounded corners, minimum width=4cm, minimum height=0.8cm,
			text centered, draw=black]
			\tikzstyle{subgroup} = [rectangle, minimum width=1.5cm, minimum height=0.6cm,
			text centered, draw=black]
			\tikzstyle{bigsubgroup} = [rectangle, minimum width=2.5cm, minimum height=0.6cm,
			text centered, draw=black]
			\tikzstyle{decision} = [diamond,minimum width=2.4cm, minimum height=1.4cm, draw=black]
			% population
			\node (evaluate) [rec] {计算适应度};
			\node (decision) at ($(evaluate.south)+(-0cm, -1.5cm)$)   [decision]
			{} node at (decision.base) {是否收敛?};
			\node at  ($(decision.south)+(0.3cm, -0.4cm)$) {否};
			\node at  ($(decision.west)+(-0.9cm, 0.2cm)$) {是};
			\node (population) at ($(decision.south)+(-0cm, -1.3cm)$) [rec]
			{种群};
			\node (end) at ($(population.west)+(-0.9cm, 0cm)$) [startstop] {结束};
			% active group
			\node (active_group_1) at ($(population.south)+(-3cm, -2.0cm)$) [subgroup]
				{活跃子群};
			\node (active_group_2) at ($(active_group_1.south)+(0cm, -1.5cm)$)
				[subgroup] {活跃子群};
			\draw[->] (decision.west) -| (end.north);
			\draw[->] (population.south) -- (active_group_1.north);
			\draw[->] (active_group_1.south) -- (active_group_2.north);
			% potential group
			\node (potential_group_1) at ($(population.south)+(0cm, -2.0cm)$) [subgroup]
				{潜在子群};
			\node (potential_group_2) at ($(potential_group_1.south)+(0cm, -1.5cm)$)
				[subgroup] {潜在子群};
			\draw[->] (evaluate.south) -- (decision.north);
			\draw[->] (decision.south) -- (population.north);
			\draw[->] (population.south) -- (potential_group_1.north);
			\draw[->] (potential_group_1.south) -- (potential_group_2.north);
			\node at ($(potential_group_1.north)+(0cm, 1.0cm)$) {分类};
			\node at ($(potential_group_2.north)+(0cm, 1.0cm)$) {选择};
			% proper group
			\node (proper_group_1) at ($(population.south)+(3cm, -2.0cm)$) [subgroup]
				{合适子群};
			\node (proper_group_2) at ($(proper_group_1.south)+(0cm, -1.5cm)$)[subgroup]
				{合适子群};
			\draw[->] (population.south) -- (proper_group_1.north);
			\draw[->] (proper_group_1.south) -- (proper_group_2.north);
			% crossover
			\node (after_cross_over) at ($(potential_group_2.south)+(0cm,
			-2.0cm)$) [rec] {后代};
			\node  at ($(after_cross_over.north)+(0cm, 1.0cm)$)  {交叉};
			\draw[->] (active_group_2.south) -- (after_cross_over.north);
			\draw[->] (potential_group_2.south) -- (after_cross_over.north);
			\draw[->] (proper_group_2.south) -- (after_cross_over.north);
			% mutation
			\node (active_group_3) at ($(after_cross_over.south)+(-2.2cm, -1.5cm)$)
				[subgroup] {活跃子群后代};
			\node at ($(after_cross_over.south)+(0cm, -1.0cm)$) {mutation};
			\draw[->] ($(after_cross_over.south)+(-1cm,0cm)$)--(active_group_3.north);
			\node (poteni_and_prop) at ($(after_cross_over.south)+(2.2cm, -1.5cm)$)
				[bigsubgroup] {潜在和合格子群后代};
			\draw[->] ($(after_cross_over.south)+(1cm,0cm)$)--(poteni_and_prop.north);

			% final draw
			\draw[->] (poteni_and_prop.east) --($(poteni_and_prop.east) + (0.2cm,0cm)$) |- (evaluate.east);
			\draw[->] (active_group_3.west) -- ($(active_group_3.west) + (-1cm,0cm)$)
				|- (evaluate.west);
		\end{tikzpicture}
	\caption{General flowchart of proposed GA model.}
	\label{fig:model}
\end{figure}

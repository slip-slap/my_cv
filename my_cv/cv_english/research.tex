%-------------------------------------------------------------------------------
%	SECTION TITLE
%-------------------------------------------------------------------------------
\cvsection{Research}

%-------------------------------------------------------------------------------
%	CONTENT
%-------------------------------------------------------------------------------
\begin{cventries}

  \cventry
    {Professor: Yokoyama Atsushi} % Role
    {Kyoto Institute of Technology} % Title
    {Japan, Kyoto} % Location
    {09/2018 - PRESENT} % Date(s)
    {
      \begin{cvitems} % Description(s)
	  \item{The multiple constrained optimization based on a variant of genetic algorithm(GA). GA is an efficient and robust algorithm to solve the optimization problem in terms of discrete variables. However, initially, GA was proposed for the unconstrained problem, to solve the constrained problem, we have to reformulate the objective function, in which append the constraints to the objective function as punishment items. To overcome the inherent drawback, we propose a new genetic algorithm with two techniques: mating pool classification and self-adaptive mutation. Then we adopt this new genetic algorithm to guide the design of a laminate under single or multiple constraints.}
	  \item{ The topology design of an artificial neural network(ANN) based on a genetic algorithm. Artificial neural networks are widely used for various scenarios, such as prediction, classification, optimization, in which the topology of an ANN plays a critical role in the performance. We propose a three-layer ANN model, in which the design variables are the number of neurons in the hidden layer, the active function, and the connection relationship. A genetic algorithm is adopted to guide the search process, and the one with the best performance is used to predict the strength of angle ply laminate. The advantage of the method is to reduce the computation cost and simplify the calculation process, compared with the traditional complicated mathematical model.  }
      \end{cvitems}
    }
    			
%---------------------------------------------------------
  \cventry
    {Professor: Yueqi Zhong} % Advisor
    {Donghua University} % Institution
    {China, Shanghai} % Location
    {2015.9-2018.3} % Date(s)
    {
      \begin{cvitems} % Description(s)
	  \item{The separation of the connected curve based on the convex hull algorithm and the elliptic curve fitting based on Fourier transformation.  Convex hull algorithm is used to obtain the smallest convex polygon which encloses a set of points. Fourier transformation can be used to remove noises on a curve. The human body data points can be obtained through a three-dimensional scan device, and the shape of the transverse section of a human body depends on the cutting point. For the transverse section at the armpit, the shape of the cross-section is a joint curve of three circles. We can use the convex hull algorithm to find the separation point. For the cross-section at the waist, the corresponding shape is a circle with noises, and Fourier transformation can remove the noise on the
			  curve.
	  }
      \end{cvitems}
    }

%---------------------------------------------------------
\end{cventries}
